\documentclass[11pt,a4paper]{article}
\usepackage[utf8]{inputenc}
\usepackage{alltt}
\usepackage{listings}
\usepackage{xcolor}
\usepackage{graphicx}
\usepackage[T1]{fontenc}
\usepackage{lmodern}
\usepackage[top=2in, bottom=1.5in, left=0.7in, right=0.7in]{geometry}

\title{Relazione Programmazione ad oggetti 2014-2015}
\author{Andrea Giacomo Baldan 579117}

\begin{document}
\maketitle
% \let\clearpage\relax
\section*{INTRODUZIONE}
Il progetto LinQedin tratta lo sviluppo di un applicazione desktop atta a gestire una rete di contatti professionali e interazioni lavorative tra utenti
basata sul famoso \textit{Social network} Linkedin\textcopyright, da cui eredita alcune funzionalità.
\section*{SPECIFICHE PROGETTUALI}
Il programma è stato sviluppato in C++/Qt 5.3.2 con compilatore gcc ver. 4.8.2 e testato su ambiente Linux Ubuntu 12.10 e 14.04, inoltre compila ed esegue con successo in ambiente Windows, testato
alla versione 7 e nei computer del laboratorio Paolotti, tuttavia utilizzando alcune funzionalità della libreria Qt rilasciate solo dalla versione 5 in poi,
la compilazione necessita del comando qmake-qt532.
Per lo sviluppo è stato utilizzato esclusivamente l'editor Sublime Text 3, non è stato utilizzato QtCreator ne QtDesigner.
\subsection*{ORGANIZZAZIONE DIRECTORY E INSTALLAZIONE}
La radice contiene il file \textit{linqedin.pro} generato con il comando qmake, \textit{main.cpp} e \textit{relazione.pdf}.
I sorgenti e gli headers della parte logica si trovano all'interno del percorso /logic, mentre /gui contiene tutti i sorgenti Qt adibiti all'interfaccia grafica.
Alcune direttive grafiche sono state inserite in un file \textit{style.qss} nella cartella /style, la directory /img contiene invece tutte le immagini necessarie all'interfaccia grafica.
Nel caso si renda necessario rigenerare il file \textit{.pro} con il comando qmake-qt532 -project si dovrà aggiungere la seguente istruzione all'interno del \textit{.pro} generato: \textit{greaterThan(QT\_MAJOR\_VERSION, 4): QT += widgets}.
\end{document}
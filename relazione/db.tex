\section*{GESTIONE DATI}
A livello logico la gestione dei dati è prerogativa esclusiva della classe \textit{LinqDB}, essa si occupa della lettura e scrittura dei dati
su file permanente in memoria, è il primo oggetto ad essere allocato all'avvio dell'applicazione e si occupa di popolare la struttura di
LinQedin in ram, rappresentata da una lista di puntatori smart alla classe base polimorfa \textit{User}, mediante metodi privati di lettura
richiamati dal metodo pubblico \textit{LinqDB::load()} e al salvataggio di eventuali modifiche mediante metodi privati di scrittura richiamati
dal metodo pubblico costante \textit{LinqDB::save()}.
La scelta di una lista come contenitore è dettata dal fatto che sia l'inserimento in coda che un eventuale rimozione risultano costanti in tempo O(1),
rispetto piuttosto ad un vector che richiederebbe il ridimensionamento per l'eliminazione di un oggetto, mentre l'utilizzo di contenitori associativi
quali set o map sarebbe risultato superfluo e inadatto allo scopo.
In caricamento o salvataggio dei dati, se non presente, un file viene automaticamente generato, l'operazione di salvataggio si occupa di sovrascrivere
i dati ad ogni chiamata del metodo, a prescindere che vi siano state eventuali modifiche o meno ai dati in ram.
\subsubsection*{Il database}
Per il salvataggio permanente dei dati in LinQedin è stato scelto di utilizzare il formato JSON, piuttosto semplice da gestire grazie alle
nuove librerie inserite nel framework Qt dalla versione 5 in poi, risulta inoltre più intuitivo e meno verboso di XML, più semplice e leggero
di un database SQlite e di facile interfacciamento per un eventuale integrazione online mediante javascript e altri linguaggi web.
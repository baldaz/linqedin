\section*{ACCOUNT ED INFO}
Ho scelto di implementare l'oggetto \textit{User} come un entità rappresentante un utente a livello astratto, estendibile mediante aggiunta di funzionalità nella gerarchia,
ciò che lo delinea sono l'account e le info associate.
\subsection*{Account}
La classe \textit{Account} è formata da credienziali d'accesso e dalle info personali dell'utente a cui è associato e da un campo
che indica il livello di privilegio del profilo utente all'interno di LinQedin, infine di un vector di puntatori alla classe \textit{Payment}.
Mantenere traccia del livello di privilegio dell'utente LinQedin all'interno del suo account semplifica alcune operazioni che avrebbero richiesto
ulteriori controlli in fase di memorizzazione e lettura dei dati su file, inoltre semplifica la gestione a livelli di astrazione successivi(es: \textit{GUI})
in quanto evita numerose operazioni di RTTI, spesso da effettuarsi solamente mediante dei \textit{dynamic\_cast} (es: generazione di differenti \textit{Widgets} in base al livello privilegio).
Le credenziali d'accesso sono rappresentate dalla classe \textit{Username}, altro non è che un oggetto formato da due campi privati di tipo \textit{string},
username e password con relativi metodi d'accesso; mentre le informazioni utente sono rappresentate da un puntatore alla classe base polimorfa astratta \textit{Info},
estesa dalle classi \textit{UserInfo} e \textit{Bio} ed estendibile in futuro per un eventuale modifica o aggiunta di altre informazioni utili ad un profilo LinQedin.
La classe \textit{Payment} è formata da un puntatore allo username del richiedente, un puntatore alla classe \textit{Subscription} che altro non è che una formalizzazione
dei piani utente proposti da LinQedin con i costi dell'offerta fissati mediante campi statici, un puntatore alla classe base \textit{BIllMethod} che contiene al momento
solamento il pagamento mediante carta di credito, ma estensibile in futuro ad altre modalità di pagamento elettronico, infine un campo booleano che indica l'approvazione o meno
del pagamento e un campo QDate che tiene traccia della data di richiesta del pagamento.
\begin{lstlisting}[language=C++]
    class Account {
    private:
        Info* _info;
        Username _user;
        privLevel _privilege;
        vector<SmartPtr<Payment> > _history;
        Avatar _avatar;
    public:
        ...
    };
\end{lstlisting}
\subsubsection*{Sistema di rappresentazione delle informazioni}
All'interno delle classi informative \textit{Info} \textit{UserInfo} e \textit{Bio} sono presenti i metodi d'accesso e modifica dei campi dati ma la logica di output
di questi ultimi è lasciata ad una classe interfaccia d'appoggio, \textit{Dispatcher}, implementata seguendo il pattern \textit{MVC}.
La classe \textit{Dispatcher} è un oggetto senza campi dati con distruttore virtuale i cui unici metodi virtuali puri che possiede ritornano una \textit{string} che rappresenta l'informazione \lq\lq formattata\rq\rq,  pronta per l'output grafico
e prendono come parametro un riferimento costante ad ogni sottotipo della classe \textit{Info}, in questo caso sono \textit{string Dispatcher::dispatch(const UserInfo\&) const = 0;} e \textit{string Dispatcher::dispatch(const Bio\&) const = 0;}.
All'interno della classe base \textit{info}, oltre ad un metodo di clonazione è stato dichiarato un metodo di stampa virtuale puro \textit{Info::dispatch} che prende come parametro
un riferimento costante ad un oggetto \textit{Dispatcher}, e nel corpo del metodo di stampa viene richiamato il metodo \textit{dispatch} presente in \textit{Dispatcher} con parametro attuale l'oggetto puntato dal puntatore \textit{this}, viene
cosi automaticamente risolto il metodo da richiamare mediante overriding e overloading utilizzando il tipo statico passato al chiamante.
In questo modo è stato possibile separare al massimo la logica di \lq\lq dispatching\rq\rq dal modello, e mediante ereditarietà è possibile creare il proprio sistema di rappresentazione dei dati; in questo caso è stata scelta una
rappresentazione in \textit{HTML} mediante la sottoclasse \textit{DispatcherHTML}.
\subsubsection*{UserInfo.cpp}
\begin{lstlisting}[language=C++]
    string UserInfo::dispatch(const Dispatcher& d) const {
        return d.dispatch(*this);
    }
\end{lstlisting}
\subsubsection*{Dispatcher.h}
\begin{lstlisting}[language=C++]
    class Dispatcher {
    public:
        virtual ~Dispatcher();
        virtual string dispatch(const UserInfo&) const = 0;
        virtual string dispatch(const Bio&) const = 0;
    };

    class DispatcherHtml : public Dispatcher{
    public:
        virtual string dispatch(const UserInfo&) const;
        virtual string dispatch(const Bio&) const;
    };
\end{lstlisting}
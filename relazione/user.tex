\section*{LOGICA}
La parte logica del progetto LinQedin è stata implementata perlopiù seguendo le linee guida offerte dal Prof. Ranzato salvo alcune personalizzazioni. La memoria
è interamente gestita mediante l'uso di puntatori smart ove necessario, negli altri casi la deallocazione avviene nei distruttori, tutti i puntatori che popolano i contenitori
utilizzati, se non associati all'uso di smart pointer, vengono prima deallocati e poi rimossi con i metodi della libreria \textit{std} \textit{erase()} e \textit{clear()}.
Ho scelto di creare una classe \textit{SmartPtr<T>}, che costituisce un puntatore smart templatizzato, per poterlo utilizzare liberamente in ogni parte del programma che richieda
gestione automatica di memoria allocata e/o per futuri scopi.
\subsection*{GERARCHIA UTENTI E RICERCA}
Un utente è formato da un account, da una rete di collegamenti e da un campo intero che funge da contavisite, possiede inoltre due classi interne
incapsulate nella parte privata di esclusivo utilizzo dell'oggetto \textit{User}, esse rappresentano due funtori utilizzati per funzionalità di
ricerca e utilità, infine è formato da due vettori contenti dei puntatori smart alla classe \textit{Messagge}, rappresentanti posta in entrata e posta in uscita;di quest'ultima, la capacità
è limitata dal grado di privilegio dell'utente.
Il campo dati account è un puntatore ad un istanza di \textit{Account} che rappresenta l'account associato all'utente, comprensivo di informazioni
personali, livello di privilegio e pagamenti dell'utente associato; mentre il campo dati \textit{net} è un puntatore ad un oggetto \textit{LinqNet},
costituito da una lista di puntatori smart alla classe base \textit{User} con metodi di aggiunta e rimozione.
\textit{User} è astratta in quanto possiede alcuni metodi virtuali puri, in particolare il metodo di ricerca, rappresenta è uno degli aspetti su cui verte maggiormente il concetto di polimorfismo, implementato mediante funtore,
ogni livello privilegio ha accesso ad un diverso grado di precisione nella ricerca. è estesa \lq\lq a cascata \rq\rq da altre 3 classi, che rappresentano
i livelli di privilegio all'interno di LinQedin.
\subsubsection*{Basic User}
Utente di livello base, derivato pubblicamente da \textit{User}, possiede funzionalità comuni a tutti gli utenti ereditate da \textit{User} come aggiunta e rimozione collegamenti dalla propria rete, \textit{getters} e \textit{setters} e i metodi di utilita' implementati in \textit{Userc.cpp}. l'override del metodo
di ricerca permette un massimo di 50 risultati e non restituisce alcuna informazione aggiuntiva sui target di ricerca se non l' effettiva
presenza all'interno del database LinQedin, l'invio dei messaggi mediante override del metodo \textit{BasicUser::sendMessage(const Username\&)} è limitato al valore del campo dati statico \textit{static int basicMailLimit}, fissato a 10.
\subsubsection*{Business User}
Un utente business possiede tutte le funzionalità di un utente basic, inoltre ha la possibilità di effettuare una ricerca più accurata ottenendo informazioni sui target visualizzando il profilo completo,
il limite è aumentato a 100 risultati, ha inoltre la facoltà di iscriversi a dei gruppi, sezioni di LinQedin amministrate da account Executive dove è possibile creare discussioni o
semplicemente dei post informativi / proposte lavorative. Implementa inoltre una funzionalità \lq\lq passiva \rq\rq, ossia una lista di utenti suggeriti da un algoritmo di
confronto che valuta la somiglianza tra account mediante alcuni criteri, quali la media pesata delle competenze, interessi e lingue, riassunta in una percentuale intera. Quest'ultima feature e' stata creata mediante l'utilizzo di un metodo di calcolo somiglianza, \textit{int similarity(const SmartPtr<User> >\&)}, che mediante medie pesate calcola una percentuale di somiglianza tra i profili in base a competenze, lingue e interessi, accoppiato ad un funtore che scorre tutto il database di LinQedin e restituisce un vettore popolato da puntatori smart ad \textit{User} che superano una certa soglia di somiglianza.
\subsubsection*{Executive User}
L'utente executive possiede le massime funzionalità all'interno di LinQedin, messaggistica in uscita illimitata, possibilità di creare ed amministrare gruppi e può visualizzare
gli ultimi 10 utenti che hanno visitato il suo profilo con allegate percentuali di somiglianza.
Può inoltre ricercare mediante filtri attivabili con il carattere \lq:\rq e unificare chiavi di ricerca multiple separandole da una virgola, per esempio ricercando: \lq:c++,perl,java\rq
restituisce una lista di utenti con competenze / interessi in c++ perl e java, fino ad un massimo di 400 risultati, ogni risultato mostra tutte le informazioni dell'utente in questione,
quali gruppi di adesione e lista collegamenti.

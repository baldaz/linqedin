\subsection*{Gerarchia Utenti}
La gerarchia degli utenti LinQedin è stata implementata perlopiù seguendo le linee guida offerte dal Prof. Ranzato salvo alcune personalizzazioni.
Un utente è formato da un account, da una rete di collegamenti e da un campo intero che funge da contavisite, possiede inoltre due classi interne
incapsulate nella parte privata di esclusivo utilizzo dell'oggetto \textit{User}, esse rappresentano due funtori utilizzati per funzionalità di
ricerca e utilità, infine è formato da due vettori contenti dei puntatori smart a messaggi, rappresentanti posta in entrata e posta in uscita, la cui capacità
è limitata dal grado di privilegio dell'utente.
Il campo dati account è un puntatore ad un istanza di \textit{Account} che rappresenta l'account associato all'utente, comprensivo di informazioni
personali, livello di privilegio e pagamenti dell'utente associato; mentre il campo dati \textit{net} è un puntatore ad un oggetto \textit{LinqNet},
costituito da una list di smartpointer a classe base \textit{User} con metodi di aggiunta e rimozione.
La classe base è astratta in quanto possiede alcuni metodi virtuali puri, è estesa \lq\lq a cascata \rq\rq da altre 3 classi, che rappresentano
i livelli di privilegio all'interno di LinQedin.
\subsubsection*{Basic User}
Utente di livello base, possiede funzionalità comuni a tutti gli utenti come aggiunta e rimozione collegamenti dalla propria rete, l'override del metodo
di ricerca permette un massimo di 50 risultati e non restituisce alcuna informazione aggiuntiva sui target di ricerca se non la loro effettiva
presenza all'interno de database LinQedin, l'invio dei messaggi mediante override del metodo \textit{BasicUser::sendMessage(const Username\&)} è limitato a 10 al mese.
\subsubsection*{Business User}
Un utente business possiede tutte le funzionalità di un utente basic, inoltre ha la possibilità di effettuare una ricerca più accurata ottenendo informazioni sui target di ricerca,
il limite è aumentato a 100 risultati e ha la facoltà di iscriversi in gruppi, sezioni di LinQedin amministrate da account Executive dove è possibile creare discussioni o
semplicemente dei post informativi / proposte lavorative. Implementa inoltre una funzionalità \lq\lq passiva \rq\rq, ossia una lista di utenti suggeriti da un algoritmo di
confronto che valuta la somiglianza tra account mediante alcuni criteri, quali la media pesata delle competenze, interessi e lingue, riassunta in una percentuale intera.
\subsubsection*{Executive User}
L'utente executive possiede le massime funzionalità all'interno di LinQedin, messaggistica in uscita illimitata, possibilità di creare ed amministrare gruppi e può visualizzare
gli ultimi 10 utenti che hanno visitato il suo profilo con allegate percentuali di somiglianza.
Può inoltre ricercare mediante filtri attivabili con il carattere \lq:\rq e unificare chiavi di ricerca multiple separandole da una virgola, per esempio ricercando: \lq:c++,perl,java\rq
restituisce una lista di utenti con competenze / interessi in c++ perl e java, fino ad un massimo di 400 risultati, ogni risultato mostra tutte le informazioni dell'utente in questione,
quali gruppi di adesione e lista collegamenti.